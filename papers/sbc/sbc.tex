\documentclass[12pt]{article}

\usepackage{sbc}
\usepackage{graphicx,url}
\usepackage[utf8]{inputenc}
\usepackage[brazil]{babel} 

\usepackage{ragged2e}
\usepackage{tikz}
\usepackage{ulem}
\usepackage{listings}
\usepackage{parcolumns}
\usepackage{amsmath,amssymb,latexsym}
\usepackage{pgf-umlsd}

\definecolor{britishracinggreen}{rgb}{0.0, 0.26, 0.15}

\tikzset{
	box/.style={draw=black!60, thick, rectangle, minimum height=4cm, minimum width=4cm}
}

\sloppy

\title{Replicação Orientada a Metaprogramação}

\author{Fellipe Augusto Ugliara\inst{1}\\
	{\normalfont Orientador:} Prof. Dr. José de Oliveira Guimarães\inst{1}\\
	{\normalfont Coorientador:} Prof. Dr. Gustavo Maciel Dias Vieira\inst{1} }


\address{Universidade Federal de São Carlos (UFSCar)\\
  Sorocaba -- SP -- Brasil \\
  Programa de Pós-Graduação em Ciência da Computação -- PPGCC-So
}

\begin{document} 

\maketitle

\begin{resumo} 
	Replicar dados é uma técnica muito poderosa usada pela computação
        distribuída. O conceito de metaprogramação pode ser aplicado 
	em diferentes contextos, neste trabalho ele está relacionado a compilação das aplicações 
	que pretendemos replicar. A metaprogramação permite que em determinadas etapas 
	de compilação o código-fonte de uma aplicação não replicada
	seja usado para gerar um código-fonte no qual essa mesma 
	aplicação seja replicável. Outras ações de validação também podem ser 
	realizadas  durante  a  geração   desse  código-fonte,  com  a
        intenção de alertar sobre possíveis 
	problemas existentes nos códigos-fonte originais. Nesse trabalho é discutida 
	uma possível solução de como validar se um código-fonte é determinista, uma 
	característica necessária do código-fonte de uma
	aplicação replicável. A contribuição desse trabalho é mostrar que a 
	metaprogramação pode auxiliar na implementação de aplicações replicadas, 
	como também instigar abordagens similares que possam ser aplicadas a 
	programação de outros requisitos não funcionais.
\end{resumo}


\begin{abstract}
	Data  replication  is  a  powerful  technique  of  distributed
        computing. The concept of metaprogramming can be implemented 
	in different contexts; in this work it is related to the compilation of applications 
	we intend to replicate. In certain compilation steps metaprogramming allows 
	the source code of a centralized application 
	to be used to generate a source code in which the same application is replicated.
	Other validation actions can also be performed during this source code 
	generation in order to warn the user about problems that possibly exist in 
	the original source code. In this work, a possible solution of how to 
	validate if a source code is deterministic is discussed, since it is a necessary 
	characteristic of the source code of a replicable application. 
	This work aims to not only show that metaprogramming can help the implementation 
	of replicated applications, but also instigate similar approaches that can be 
	employed in the programming of other non-functional requirements.
\end{abstract}

\section{Introdução e Motivação} 

A implementação de replicação em uma aplicação tem como objetivo melhorar a 
confiabilidade, tolerância a falhas e acessibilidade dessa aplicação.
O requisito de confiabilidade define que ela estará disponível a 
qualquer momento que um usuário precise utiliza-lá, tolerância a falhas garante 
que caso a aplicação sofra um mal súbito ela poderá retomar a execução do ponto 
onde esse problema ocorreu, e acessibilidade define que um usuário poderá acessar 
a aplicação de qualquer local que for necessário. 

A replicação é um requisito não funcional que resolve problemas referentes
a tecnologia e não ao domínio para o qual uma aplicação é desenvolvida. Os mesmos 
requisitos não funcionais pode ser atribuídos a aplicações de domínios diferentes.
Uma aplicação financeira pode implementar replicação para atender aos requisitos
mencionados, assim como uma aplicação de comercio eletrônico pode usar dessa
mesma implementação de replicação para atender a esses mesmos requisitos não
funcionais.  
 
Como a replicação não é a motivação primária do desenvolvimento de uma aplicação,
o desejo é que a implementação da replicação seja o mais coeso e menos acoplado 
possível com o restante do código-fonte implementado. A coesão da implementação
é uma métrica que define se as partes (módulos, componentes) do código-fonte de 
uma aplicação tratam cada uma de um único assunto (problema, objetivo, funcionalidade). 
O acoplamento é uma métrica usada para definir o quanto dependente são essas partes 
do código-fonte. Para que essas partes de código-fonte sejam reutilizáveis e de 
fácil manutenção elas devem ser mais coesas e menos acopladas.

\begin{center}
\begin{tikzpicture}[yscale=-1,scale=1.3]

\node[anchor=north, align=left, font=\small] at (4.5,1) {Código-Fonte\\- Replicação\\- Metaprogramação};
\node[anchor=north, align=left, font=\small] at (8.5,1) {Código-Fonte\\+ Replicação\\- Metaprogramação};
\node[anchor=north, align=left, font=\small] at (12.4,1) {Código-Fonte\\+ Replicação\\+ Metaprogramação};

\begin{scope}[shift={(0,1.5)}]
\filldraw[box, fill=red!20] (4,2) circle (0.8cm);
\filldraw[box, fill=red!20] (8.3,2) circle (0.8cm);
\filldraw[box, fill=red!20] (12.4,2) circle (0.8cm);

\filldraw[box, fill=yellow!20, fill opacity=0.7] (13.3,2.4) circle (0.4cm);
\filldraw[box, fill=yellow!20, fill opacity=0.7] (8.6,2.4) circle (0.4cm);
\end{scope}

\node[anchor=north, align=left, font=\small] at (3.9,4.5) {+ Coeso\\- Acoplado};
\node[anchor=north, align=left, font=\small] at (8.5,4.5) {- Coeso\\+ Acoplado};
\node[anchor=north, align=left, font=\small] at (12.5,4.5) {+ Coeso\\- Acoplado};

\begin{scope}[shift={(2.15,1.5)}]
\begin{scope}[shift={(0,0)}]
\filldraw[box, fill=red!20] (1,4.5) rectangle (1.25,4.75);
\node[anchor=west] at (1.2,4.65) {Programa};
\end{scope}
\begin{scope}[shift={(2.5,0)}]
\filldraw[box, fill=yellow!20] (1,4.5) rectangle (1.25,4.75);
\node[anchor=west] at (1.2,4.65) {Replicação};
\end{scope}
\end{scope}

\end{tikzpicture}
\end{center}

A motivação desse trabalho é mostrar como a metaprogramação em compilação pode 
auxiliar na implementação de aplicações replicáveis, tornando os códigos-fonte  
dessas implementações mais coesos e menos acoplados~\cite{fellipe:2018}. O prefixo 
`meta' em metaprogramação
define uma referencia a si mesmo, como em metadados que são os dados que contêm 
informação a respeito dos próprios dados. Metaprogramação é o conceito que permite 
definir programas que referenciam entidades da própria programação (compiladores, 
interpretadores, códigos-fonte, programas, elementos da linguagem utilizada)
\cite{taxonomy:2015}, para definir como esses programas devem ser 
trabalhados por essas entidades.

Neste trabalho, usando metaprogramação em compilação, foi possível desenvolver 
programas sem replicação que durante a compilação são usados como entrada para 
gerar novos programas com replicação~\cite{metaobject:2018}. 
Esses novos programas também mantêm as
funcionalidades dos programas originais. Na compilação  foi possível verificar
esses programas na busca de trechos de código-fonte não deterministas que possam 
levar as aplicações replicadas a apresentarem problemas durante a 
execução.

O determinismo em aplicações replicadas é necessário porque cada ação gerada em 
uma réplica dessa aplicação será compartilhada com as demais réplicas que estiverem 
vinculadas a um mesmo grupo de execução. Caso essas ações não sejam deterministas 
as réplicas poderão assumir comportamentos diferentes o que tornaria a execução 
desse grupo de réplicas inconsistente~\cite{vieira:2010}. 
Um grupo de execução é um conjunto de 
réplicas, de uma mesma aplicação, que estão vinculadas a uma mesma execução. 

Todas as réplicas de um grupo de execução serão iguais, assim se uma réplica 
sofrer um mal súbito ela poderá ser recriada a partir de outra réplica que esteja 
no mesmo grupo, o que melhora a tolerância a falhas das aplicações. As requisições 
dos usuários de aplicações replicadas podem ser distribuídas entre as réplicas 
para melhorar a confiabilidade. As réplicas também podem ser executadas em pontos 
geográficos diferentes para melhorar a acessibilidade dessas aplicações.

\begin{center}
\begin{tikzpicture}[yscale=-1,scale=1.35]
%\draw [gray,very thin] (0,0) grid (10.8,4);

\node[anchor=north] at (1.5,0.15) {\uline{Programa Não Replicado}};

\node[anchor=north] at (1.5,1) {Instância Única};

\begin{scope}[scale=0.8, shift={(0.38,0.5)}]
\filldraw[box, fill=red!20] (1,1.5) rectangle (2,2.5);
\filldraw[box, fill=blue!20] (1,2) rectangle (1.5,2.5);
\end{scope}

\begin{scope}
\draw[->, thick, draw=black!70] (0,3.5) -- (3,3.5);
\filldraw[fill=black!70, draw=black!70] (0.05,3.5) circle (1.5pt);
\end{scope}

\filldraw[fill=black!70, draw=black!70] (1.3,2.2) circle (1.5pt);

\begin{scope}
\draw[->, thick, draw=black!70] (1.3,2.2) -- (1.3,3.4);
\node[anchor=north west] at (1.25,2.65) {Execução};
\filldraw[fill=blue!20, draw=black!70] (1.3,3.5) circle (2pt);
\end{scope}

% ----------------------------------------------------
\draw[thick, dash dot, draw=black!70] (3.92,0) -- (3.92,4.12);

\node[anchor=north] at (7.1,0.15) {\uline{Programa Replicado}};

\node[anchor=north] at (5.2,1) {Réplica A};

\begin{scope}[scale=0.8, shift={(5,0.5)}]
\filldraw[box, fill=red!20] (1,1.5) rectangle (2,2.5);
\filldraw[box, fill=blue!20] (1,2) rectangle (1.5,2.5);
\filldraw[box, fill=yellow!20] (1.5, 2) rectangle (2,2.5);
\end{scope}

\node[anchor=north] at (7.2,1) {Réplica B};

\begin{scope}[scale=0.8, shift={(7.5,0.5)}]
\filldraw[box, fill=red!20] (1,1.5) rectangle (2,2.5);
\filldraw[box, fill=blue!20] (1,2) rectangle (1.5,2.5);
\filldraw[box, fill=yellow!20] (1.5, 2) rectangle (2,2.5);
\end{scope}

\node[anchor=north] at (9.2,1) {Réplica C};

\begin{scope}[scale=0.8, shift={(10,0.5)}]
\filldraw[box, fill=red!20] (1,1.5) rectangle (2,2.5);
\filldraw[box, fill=blue!20] (1,2) rectangle (1.5,2.5);
\filldraw[box, fill=yellow!20] (1.5, 2) rectangle (2,2.5);
\end{scope}

\begin{scope}[shift={(4.6,0.0)}]
\draw[->, thick, draw=black!70] (0,3.5) -- (5.3,3.5);
\filldraw[fill=black!70, draw=black!70] (0.05,3.5) circle (1.5pt);
\end{scope}

\filldraw[fill=black!70, draw=black!70] (5.4,2.2) circle (1.5pt);
\filldraw[fill=black!70, draw=black!70] (7.4,2.2) circle (1.5pt);
\filldraw[fill=black!70, draw=black!70] (9.4,2.2) circle (1.5pt);

\begin{scope}[shift={(4.1,0.0)}]
\draw[thick, draw=black!70] (1.3,2.2) -- (1.3,2.75);
\node[anchor=north east] at (2.1,2.65) {Requisição};
\draw[->, thick, draw=black!70] (1.3,3.1) -- (1.3,3.4);
\filldraw[fill=yellow!20, draw=black!70] (1.3,3.5) circle (2pt);
\end{scope}

\begin{scope}[shift={(4.5,0.0)}]
\node[anchor=north east] at (3.1,3.5) {Consenso};
\draw[->, thick, draw=black!70] (0.9,2.2) to[bend right]  (1.99,3.4);
\filldraw[fill=yellow!20, draw=black!70] (2,3.5) circle (2pt);
\draw[->, thick, draw=black!70] (2.9,2.2) to[bend left]  (2.30,3.4);
\filldraw[fill=yellow!20, draw=black!70] (2.3,3.5) circle (2pt);
\draw[->, thick, draw=black!70] (4.9,2.2) to[bend left=55]  (2.63,3.4);
\filldraw[fill=yellow!20, draw=black!70] (2.6,3.5) circle (2pt);
\end{scope}

\filldraw[fill=black!70, draw=black!70] (5,2.2) circle (1.5pt);
\filldraw[fill=black!70, draw=black!70] (7,2.2) circle (1.5pt);
\filldraw[fill=black!70, draw=black!70] (9,2.2) circle (1.5pt);

\begin{scope}[shift={(6.5,0.0)}]
\node[anchor=north east] at (3.1,3.5) {Execução};
\draw[->, thick, draw=black!70] (-1.5,2.2) to[bend right] (1.97,3.4);
\filldraw[fill=blue!20, draw=black!70] (2,3.5) circle (2pt);
\draw[->, thick, draw=black!70] (0.5,2.2) to[bend right=55] (2.28,3.4);
\filldraw[fill=blue!20, draw=black!70] (2.3,3.5) circle (2pt);
\draw[->, thick, draw=black!70] (2.5,2.2) to[bend left] (2.58,3.4);
\filldraw[fill=blue!20, draw=black!70] (2.6,3.5) circle (2pt);
\end{scope}

\begin{scope}[shift={(-1.2,0)}]
\begin{scope}[shift={(0,0)}]
\filldraw[box, fill=red!20] (1,4.5) rectangle (1.25,4.75);
\node[anchor=west] at (1.2,4.65) {Programa};
\end{scope}
\begin{scope}[shift={(2.25,0)}]
\filldraw[box, fill=blue!20] (1,4.5) rectangle (1.25,4.75);
\node[anchor=west] at (1.2,4.65) {Ação};
\end{scope}
\begin{scope}[shift={(3.8,0)}]
\filldraw[box, fill=yellow!20] (1,4.5) rectangle (1.25,4.75);
\node[anchor=west] at (1.2,4.65) {Replicação};
\end{scope}
\end{scope}

\end{tikzpicture}
\end{center}

Desenvolver aplicações replicadas exige uma certa habilidade por parte da equipe de 
desenvolvimento para que a aplicação seja determinista, também é necessário que 
bibliotecas ou \textit{frameworks} que realizam a replicação sejam estudados o que 
pode exigir tempo e maturidade dessa equipe.
Usar metaprogramação nesse cenário melhora o desenvolvimento, pois torna a replicação
transparente para a equipe de desenvolvimento, o que reduz a necessidade de compreender 
como essas bibliotecas ou \textit{frameworks} funcionam. A metaprogramação também 
melhora o tempo gasto em desenvolvimento que se torna menor uma vez que a 
responsabilidade de tornar a aplicação replicável fica a cargo da metaprogramação e 
não da equipe de desenvolvimento.

\section{Replicação e Metaprogramação}

Esse documento é um resumo da dissertação \textit{Replicação Orientada a Metaprogramação}
\cite{fellipe:2018}. Nessa dissertação a abordagem explicada 
na sessão anterior foi desenvolvida e validada, usando o \textit{framework} Treplica
\cite{vieira:2008} para prover replicação e a linguagem de programação Cyan
\cite{cyan:2018} por permitir metaprogramação em compilação. Aqui será apresentado 
um resumo desse trabalho, e na próxima sessão as conclusões e impactos da pesquisa.

Treplica é um \textit{framework} baseado em estados e ações. As aplicações 
desenvolvidas com ele implementam seu comportamento no formato de ações que 
quando passados à máquina de estados de Treplica,  
alteram seu estado interno. Essas ações quando enviadas da aplicação para máquina 
de estados de Treplica também são repassadas para o grupo de réplicas 
a qual a réplica que enviou a ação está vinculada. Essas ações quando recebidas 
pelas demais réplicas modificam o estado interno da máquina de estados de Treplica
de quem a recebeu. Desse modo todas as réplicas de um grupo sempre estarão com a
máquina de Treplica no mesmo estado, essas réplicas serão iguais.

A metaprogramação em Cyan é aplicada durante a compilação das aplicações. Nessa 
linguagem é possível desenvolver programas (metaprogramas) que são executados pelo 
compilador de Cyan no momento da compilação de um código-fonte. Com base em 
elementos encontrados no código-fonte que está sendo compilado esses 
metaprogramas são executados pelo compilador para gerar uma nova versão do 
código-fonte original. Os metaprogramas 
usados pelo compilador de Cyan devem ser implementados usando o protocolo 
de metaobjetos (biblioteca) fornecido junto ao código-fonte do compilador 
da linguagem Cyan.

Nessa pesquisa foi desenvolvido um metaprograma em Cyan que durante a compilação
gera uma versão replicável de aplicações que não implementam replicação.
Com base em marcações (anotações) encontradas no código-fonte dessas aplicações 
não replicadas o compilador chama os metaprogramas desenvolvidos para
gerar uma versão replicável da mesma aplicação. Essa nova versão usa o 
\textit{framework} Treplica para implementar a replicação nessas aplicações.

\begin{center}
\begin{tikzpicture}[yscale=-1,scale=1.218]

\node[draw=black!60, fill=red!20, thick, anchor=north west, align=center] at (0,1) {Código-Fonte Cyan\\com @anotações};
\node[draw=black!60, fill=green!20, thick, anchor=north west, align=center] at (3.9,1) {Metaoprogramas das\\@anotações};
\node[draw=black!60, fill=gray!20, thick, anchor=north west, align=center] at (0,3) {Compilação\\Etapa A};
\node[draw=black!60, fill=red!20, thick, anchor=north west, align=center] at (4,3) {Nova versão do\\Código-Fonte Cyan};
\node[draw=black!60, fill=gray!20, thick, anchor=north west, align=center] at (8.78,3) {Compilação\\Etapa B};
\node[draw=black!60, fill=red!20, thick, anchor=north west, align=center] at (8.5,5) {Código-Fonte\\Final};

\draw[->, thick, draw=black!70] (0.95,2.05) to (0.95, 2.98);
\draw[->, thick, draw=black!70] (5.5,2.05) .. controls (5.5,3) and (1.05,2) .. (1.1, 2.98);
\draw[->, thick, draw=black!70] (1.95,3.5) to (3.98, 3.5);
\draw[->, thick, draw=black!70] (7.08,3.5) to (8.77, 3.5);
\draw[->, thick, draw=black!70] (9.7,4.05) to (9.7, 4.99);

\begin{scope}[shift={(-0.9,1.15)}]
\begin{scope}[shift={(0,0)}]
\filldraw[box, fill=red!20] (1,4.5) rectangle (1.25,4.75);
\node[anchor=west] at (1.2,4.65) {Código-Fonte};
\end{scope}
\begin{scope}[shift={(2.95,0)}]
\filldraw[box, fill=green!20] (0.85,4.5) rectangle (1.1,4.75);
\node[anchor=west] at (1.1,4.65) {Metaprograma};
\end{scope}
\begin{scope}[shift={(5.7,0)}]
\filldraw[box, fill=gray!20] (1,4.5) rectangle (1.25,4.75);
\node[anchor=west] at (1.2,4.65) {Compilador};
\end{scope}
\end{scope}

\end{tikzpicture}
\end{center}

Outro metaprograma que foi desenvolvido nessa pesquisa valida se o código-fonte
não replicado apresenta casos de não determinismo. Essa validação não
cobriu todos as possíveis situações de não determinismo e necessita de uma 
configuração prévia para poder identificar essas ocorrências. Essa configuração
mapeia quais métodos podem ser considerados não deterministas,
com base nessa lista o metaprograma visita os métodos implementados no código-fonte 
da aplicação procurando por chamadas a essas métodos não 
deterministas listados.

Detalhes de implementação como códigos-fonte exemplo podem ser encontrados 
no texto da dissertação \cite{fellipe:2018}. Nela também são mostrados outros \textit{frameworks}
usados para implementar replicação e outras linguagens de programação que 
tem suporte a outras formas de metaprogramação. 

\section{Contribuição e Trabalhos Futuros }

Nesse trabalho foi proposto o uso da metaprogramação para permitir a implementação 
de requisitos não funcionais de uma aplicação de modo mais transparente. Essa 
proposta foi validada no contexto de aplicações replicáveis. Trabalhos futuros podem
usar a mesma proposta em contextos diferentes como segurança, portabilidade, 
desempenho, usabilidade, padrões de código-fonte, e outros.

A validação aplicada ao caso de não determinismo mostra uma outra forma possível de 
utilizar a metaprogramação. Ela pode ser usada no intuito de verificar o código-fonte 
das aplicações na busca de erros conceituais de programação, que não seriam indicados
pelo processo padrão de compilação. Essa abordagem também pode ser aplicada a problemas 
de outras áreas da computação, para identificar possíveis erros no uso de bibliotecas
e \textit{framework}.

A metaprogramação em compilação é um conceito que pode ser usado para 
padronizar atividades de programação realizadas de modo manual por equipes de 
desenvolvimento. Muitas implementações e verificações que hoje são realizadas de forma 
manual podem ser automatizadas em linguagens que permitem esse tipo de metaprogramação.

Durante a realização dessa pesquisa foi submetido um artigo, \textit{Transparent Replication 
Using Metaprogramming in Cyan} \cite{sblp:2018}, ao Simpósio Brasileiro de 
Linguagens de Programação (SBLP), que aconteceu em Fortaleza (Ceará, Brasil).
Nesse artigo é mostrada a automatização do desenvolvimento do código-fonte. Ele não
trata da verificação desse código-fonte. A automatização foi aplicada em programas
distribuídos que foram desenvolvidos em Cyan usando Treplica. Essa publicação do ponto de vista 
dos detalhes técnicos está situada entre esse artigo a dissertação 
\textit{Replicação Orientada a Metaprogramação} \cite{fellipe:2018}.

O código-fonte desenvolvido nessa pesquisa está disponível na internet em um
repositório de projetos (https://bitbucket.org/fellipe-ugliara/mestrado/src/master/). 
Caso seja necessário reproduzir essa pesquisa, o ponto de partida
é esse repositório. Para os entusiastas do \LaTeX, também estão disponíveis: o código-fonte
dessa dissertação, e o código-fonte da apresentação usada na defesa desse trabalho. Os
arquivos README.md do repositório contêm detalhes de como executar esse projeto.

\bibliographystyle{sbc}
\bibliography{sbc}

\end{document}
