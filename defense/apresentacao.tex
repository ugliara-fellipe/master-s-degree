\documentclass[10pt]{beamer}

\usetheme[progressbar=frametitle]{metropolis}
\usepackage{appendixnumberbeamer}

\usepackage{booktabs}
\usepackage[scale=2]{ccicons}

\usepackage[brazilian]{babel}
\usepackage[utf8]{inputenc}
\usepackage[T1]{fontenc}

\usepackage{ragged2e}
\usepackage{tikz}
\usepackage{ulem}
\usepackage{listings}
\usepackage{parcolumns}
\usepackage{amsmath,amssymb,latexsym}
\usepackage{pgf-umlsd}

\definecolor{britishracinggreen}{rgb}{0.0, 0.26, 0.15}

\tikzset{
box/.style={draw=black!60, thick, rectangle, minimum height=4cm, minimum width=4cm}
}

\usepackage{pgfplots}

\usepackage{xspace}

\title{Replicação Orientada a Metaprogramação}
\subtitle{Apresentação de Defesa de Mestrado}
\date{12 de Junho de 2018}
\author{Fellipe Augusto Ugliara}
\institute{Universidade Federal de São Carlos – UFSCar}

\begin{document}

\maketitle

% ==============================================================

\begin{frame}{Introdução}

\justifying
A \alert{\textbf{metaprogramação}}, da linguagem de programação \alert{\textbf{Cyan}}, foi usada para tornam o código-fonte dos programas \alert{\textbf{replicados}}, desenvolvidos usam \hbox{\alert{\textbf{Treplica}}},  mais \alert{\textbf{coesos}} e menos \alert{\textbf{acoplados}}.

\vspace{15px}

\begin{tikzpicture}[yscale=-1]

\node[anchor=north, align=left, font=\small] at (4.5,1) {Código-Fonte\\- Replicação\\- Metaprogramação};
\node[anchor=north, align=left, font=\small] at (8.5,1) {Código-Fonte\\+ Replicação\\- Metaprogramação};
\node[anchor=north, align=left, font=\small] at (12.4,1) {Código-Fonte\\+ Replicação\\+ Metaprogramação};

\begin{scope}[shift={(0,1.5)}]
\filldraw[box, fill=red!20] (4,2) circle (0.8cm);
\filldraw[box, fill=red!20] (8.3,2) circle (0.8cm);
\filldraw[box, fill=red!20] (12.4,2) circle (0.8cm);

\filldraw[box, fill=yellow!20, fill opacity=0.7] (13.3,2.4) circle (0.4cm);
\filldraw[box, fill=yellow!20, fill opacity=0.7] (8.6,2.4) circle (0.4cm);
\end{scope}

\node[anchor=north, align=left, font=\small] at (3.9,4.5) {+ Coeso\\- Acoplado};
\node[anchor=north, align=left, font=\small] at (8.5,4.5) {- Coeso\\+ Acoplado};
\node[anchor=north, align=left, font=\small] at (12.5,4.5) {+ Coeso\\- Acoplado};

\begin{scope}[shift={(2.15,1.5)}]
\begin{scope}[shift={(0,0)}]
  \filldraw[box, fill=red!20] (1,4.5) rectangle (1.25,4.75);
  \node[anchor=west] at (1.2,4.65) {Programa};
\end{scope}
\begin{scope}[shift={(2.5,0)}]
  \filldraw[box, fill=yellow!20] (1,4.5) rectangle (1.25,4.75);
  \node[anchor=west] at (1.2,4.65) {Replicação};
\end{scope}
\end{scope}

\end{tikzpicture}

\end{frame}

% ==============================================================

\begin{frame}{Replicação}

\justifying
Replicação é um método usado na implementação de programas de computador que podem ser tolerantes a falhas.

\vspace{15px}

\begin{tikzpicture}[yscale=-1]
%\draw [gray,very thin] (0,0) grid (10.8,4);

\node[anchor=north] at (1.5,0.15) {\uline{Programa Não Replicado}};

\node[anchor=north] at (1.5,1) {Instância Única};

\begin{scope}[scale=0.8, shift={(0.38,0.5)}]
  \filldraw[box, fill=red!20] (1,1.5) rectangle (2,2.5);
  \filldraw[box, fill=blue!20] (1,2) rectangle (1.5,2.5);
\end{scope}

\begin{scope}
  \draw[->, thick, draw=black!70] (0,3.5) -- (3,3.5);
  \filldraw[fill=black!70, draw=black!70] (0.05,3.5) circle (1.5pt);
\end{scope}

\filldraw[fill=black!70, draw=black!70] (1.3,2.2) circle (1.5pt);

\begin{scope}
  \draw[->, thick, draw=black!70] (1.3,2.2) -- (1.3,3.4);
  \node[anchor=north west] at (1.25,2.65) {Execução};
  \filldraw[fill=blue!20, draw=black!70] (1.3,3.5) circle (2pt);
\end{scope}

% ----------------------------------------------------
\draw[thick, dash dot, draw=black!70] (3.92,0) -- (3.92,4.12);

\node[anchor=north] at (7.1,0.15) {\uline{Programa Replicado}};

\node[anchor=north] at (5.2,1) {Réplica A};

\begin{scope}[scale=0.8, shift={(5,0.5)}]
  \filldraw[box, fill=red!20] (1,1.5) rectangle (2,2.5);
  \filldraw[box, fill=blue!20] (1,2) rectangle (1.5,2.5);
  \filldraw[box, fill=yellow!20] (1.5, 2) rectangle (2,2.5);
\end{scope}

\node[anchor=north] at (7.2,1) {Réplica B};

\begin{scope}[scale=0.8, shift={(7.5,0.5)}]
  \filldraw[box, fill=red!20] (1,1.5) rectangle (2,2.5);
  \filldraw[box, fill=blue!20] (1,2) rectangle (1.5,2.5);
  \filldraw[box, fill=yellow!20] (1.5, 2) rectangle (2,2.5);
\end{scope}

\node[anchor=north] at (9.2,1) {Réplica C};

\begin{scope}[scale=0.8, shift={(10,0.5)}]
  \filldraw[box, fill=red!20] (1,1.5) rectangle (2,2.5);
  \filldraw[box, fill=blue!20] (1,2) rectangle (1.5,2.5);
  \filldraw[box, fill=yellow!20] (1.5, 2) rectangle (2,2.5);
\end{scope}

\begin{scope}[shift={(4.6,0.0)}]
  \draw[->, thick, draw=black!70] (0,3.5) -- (5.3,3.5);
  \filldraw[fill=black!70, draw=black!70] (0.05,3.5) circle (1.5pt);
\end{scope}

\filldraw[fill=black!70, draw=black!70] (5.4,2.2) circle (1.5pt);
\filldraw[fill=black!70, draw=black!70] (7.4,2.2) circle (1.5pt);
\filldraw[fill=black!70, draw=black!70] (9.4,2.2) circle (1.5pt);

\begin{scope}[shift={(4.1,0.0)}]
  \draw[thick, draw=black!70] (1.3,2.2) -- (1.3,2.75);
  \node[anchor=north east] at (2.1,2.65) {Requisição};
  \draw[->, thick, draw=black!70] (1.3,3.1) -- (1.3,3.4);
  \filldraw[fill=yellow!20, draw=black!70] (1.3,3.5) circle (2pt);
\end{scope}

\begin{scope}[shift={(4.5,0.0)}]
  \node[anchor=north east] at (3.1,3.5) {Consenso};
  \draw[->, thick, draw=black!70] (0.9,2.2) to[bend right]  (1.99,3.4);
  \filldraw[fill=yellow!20, draw=black!70] (2,3.5) circle (2pt);
  \draw[->, thick, draw=black!70] (2.9,2.2) to[bend left]  (2.30,3.4);
  \filldraw[fill=yellow!20, draw=black!70] (2.3,3.5) circle (2pt);
  \draw[->, thick, draw=black!70] (4.9,2.2) to[bend left=55]  (2.63,3.4);
  \filldraw[fill=yellow!20, draw=black!70] (2.6,3.5) circle (2pt);
\end{scope}

\filldraw[fill=black!70, draw=black!70] (5,2.2) circle (1.5pt);
\filldraw[fill=black!70, draw=black!70] (7,2.2) circle (1.5pt);
\filldraw[fill=black!70, draw=black!70] (9,2.2) circle (1.5pt);

\begin{scope}[shift={(6.5,0.0)}]
  \node[anchor=north east] at (3.1,3.5) {Execução};
  \draw[->, thick, draw=black!70] (-1.5,2.2) to[bend right] (1.97,3.4);
  \filldraw[fill=blue!20, draw=black!70] (2,3.5) circle (2pt);
  \draw[->, thick, draw=black!70] (0.5,2.2) to[bend right=55] (2.28,3.4);
  \filldraw[fill=blue!20, draw=black!70] (2.3,3.5) circle (2pt);
  \draw[->, thick, draw=black!70] (2.5,2.2) to[bend left] (2.58,3.4);
  \filldraw[fill=blue!20, draw=black!70] (2.6,3.5) circle (2pt);
\end{scope}

\begin{scope}[shift={(-1.2,0)}]
\begin{scope}[shift={(0,0)}]
  \filldraw[box, fill=red!20] (1,4.5) rectangle (1.25,4.75);
  \node[anchor=west] at (1.2,4.65) {Programa};
\end{scope}
\begin{scope}[shift={(2.25,0)}]
  \filldraw[box, fill=blue!20] (1,4.5) rectangle (1.25,4.75);
  \node[anchor=west] at (1.2,4.65) {Ação};
\end{scope}
\begin{scope}[shift={(3.8,0)}]
  \filldraw[box, fill=yellow!20] (1,4.5) rectangle (1.25,4.75);
  \node[anchor=west] at (1.2,4.65) {Replicação};
\end{scope}
\end{scope}

\end{tikzpicture}

\end{frame}

% ==============================================================

\begin{frame}{Programa em Cyan}

\begin{tikzpicture}[yscale=-1]
\begin{scope}[shift={(0,1)}]

\begin{scope}[shift={(4.385,0.015)}]
  \filldraw[draw=black!60, thick, fill=red!20] (0,0) -- (0.8,0) -- (0.8,0.8) -- (0,0) -- cycle;
\end{scope}

\begin{scope}
\node[draw=black!60, thick, anchor=north west, align=left] at (0,0) {
package main \\
\\
object Program \\
~~func run: Array<String> args \{ \\
~~~~var info = Info new; \\
~~~~if args[1] == "ReplicaA" \{ \\
~~~~~~info setText: "text"; \\
~~~~\} \\
~~\} \\
end
};
\end{scope}

\begin{scope}[shift={(10.01,0.015)}]
  \filldraw[draw=black!60, thick, fill=red!20] (0,0) -- (0.8,0) -- (0.8,0.8) -- (0,0) -- cycle;
\end{scope}

\begin{scope}[shift={(6.3,0)}]
\node[draw=black!60, thick, anchor=north west, align=left] at (0,0) {
package main \\
\\
object Info \\
~~var String text \\
~~func init \{ self.text = "~";\} \\
\\
~~func setText: String text \{ \\
~~~~self.text = text; \\
~~\} \\
\\
~~func getText -> String \{ \\
~~~~return self.text; \\
~~\} \\
end
};
\end{scope}

\begin{scope}[scale=0.8, shift={(1.8,5)}]
  \filldraw[box, fill=red!20] (1,1.5) rectangle (2,2.5);
  \filldraw[box, fill=blue!20] (1,2) rectangle (1.5,2.5);
\end{scope}

\filldraw[fill=black!70, draw=black!70] (2.45,5.8) circle (1.5pt);

\draw[very thick, draw=blue!45] (0.7,3.4) to (3.9,3.4);
\draw[->, thick, draw=black!70] (2.45,5.8) to[bend right=55] (1,3.5);

\draw[very thick, draw=blue!45] (6.55,2.9) to (6.55,4.5);
\draw[->, thick, draw=black!70] (2.45,5.8) to[bend right] (6.5, 3.25);

\begin{scope}[shift={(-0.8,2)}]
\begin{scope}[shift={(0,0)}]
  \filldraw[box, fill=red!20] (1,4.5) rectangle (1.25,4.75);
  \node[anchor=west] at (1.2,4.65) {Programa};
\end{scope}
\begin{scope}[shift={(2.25,0)}]
  \filldraw[box, fill=blue!20] (1,4.5) rectangle (1.25,4.75);
  \node[anchor=west] at (1.2,4.65) {Ação};
\end{scope}
\end{scope}

\end{scope}
\end{tikzpicture}

\end{frame}

% ==============================================================

\begin{frame}{Programa Usando Treplica}

\begin{tikzpicture}[yscale=-1]
\begin{scope}[shift={(0,1)}]

\begin{scope}[shift={(4.42,0.018)}]
  \filldraw[draw=black!60, thick, fill=red!20] (0,0) -- (0.8,0) -- (0.8,0.8) -- (0,0) -- cycle;
\end{scope}

\begin{scope}
\node[draw=black!60, thick, anchor=north west, align=left, font=\scriptsize] at (0,0) {
package main \\
import treplica \\
\\
object Program \\
~~func run: Array<String> args \{ \\
~~~~var info = Info new; \\
~~~~var treplica = Treplica new; \\
~~~~treplica runMachine: \\
~~~~~~info numberProcess: 3 rtt: 200 \\
~~~~~~path: "/tmp" ++ args[1]; \\
~~~~var action = UpdateAction new: "text"; \\
~~~~if args[1] == "ReplicaA" \{ \\
~~~~~~treplica execute : action ; \\
~~~~\} \\
~~\} \\
end
};
\end{scope}

\begin{scope}[shift={(10.09,5.015)}]
  \filldraw[draw=black!60, thick, fill=red!20] (0,0) -- (0.8,0) -- (0.8,0.8) -- (0,0) -- cycle;
\end{scope}

\begin{scope}[shift={(7.05,5)}]
\node[draw=black!60, thick, anchor=north west, align=left, font=\scriptsize] at (0,0) {
package main \\
import treplica \\
\\
object Info extends Context \\
~~... \\
~~func setText: String text \{...\} \\
~~... \\
end
};
\end{scope}

\begin{scope}[shift={(10.05,0.019)}]
  \filldraw[draw=black!60, thick, fill=blue!20] (0,0) -- (0.8,0) -- (0.8,0.8) -- (0,0) -- cycle;
  \filldraw[draw=black!60, thick, fill=red!20] (0,0) -- (0.5,0) -- (0.5,0.5) -- (0,0) -- cycle;
\end{scope}

\begin{scope}[shift={(6.32,0)}]
\node[draw=black!60, thick, anchor=north west, align=left, font=\scriptsize] at (0,0) {
package main \\
import treplica \\
\\
object UpdateAction extends Action \\
~~var String updateText \\
~~func init: String text \{ \\
~~~~self.updateText = text; \\
~~\} \\
\\
~~func executeOn: Context context ~\{ \\
~~~~var info = Info cast: context; \\
~~~~info setText: self.updateText; \\
~~\} \\
end
};
\end{scope}

\begin{scope}[scale=0.8, shift={(1.8,5.75)}]
  \filldraw[box, fill=red!20] (1,1.5) rectangle (2,2.5);
  \filldraw[box, fill=blue!20] (1,2) rectangle (1.5,2.5);
  \filldraw[box, fill=yellow!20] (1.5, 2) rectangle (2,2.5);
\end{scope}

\filldraw[fill=black!70, draw=black!70] (2.85,6.4) circle (1.5pt);
\filldraw[fill=black!70, draw=black!70] (2.45,6.4) circle (1.5pt);

\draw[very thick, draw=orange!47] (0.35,2) to (0.35,3.25);
\draw[->, thick, draw=black!70] (2.85,6.4) to[bend right=47] (0.34,3.3);

\draw[very thick, draw=orange!47] (8.4,6.35) to (10.25,6.35);
\draw[->, thick, draw=black!70] (2.85,6.4) to[bend right=17] (9,6);

\draw[very thick, draw=blue!45] (0.6,4.3) to (3.6,4.3);
\draw[->, thick, draw=black!70] (2.45,6.4) to (2.45, 4.4);

\draw[very thick, draw=blue!45] (6.8,4) to (10.2,4);
\draw[->, thick, draw=black!70] (2.45,6.4) to[bend right] (6.6, 4);

\draw[very thick, draw=blue!45] (7.3,7.05) to (10.75,7.05);
\draw[->, thick, draw=black!70] (2.45,6.4) to[bend left=20] (7.25, 7.05);

\begin{scope}[shift={(-0.9,2.9)}]
\begin{scope}[shift={(0,0)}]
  \filldraw[box, fill=red!20] (1,4.5) rectangle (1.25,4.75);
  \node[anchor=west] at (1.2,4.65) {Programa};
\end{scope}
\begin{scope}[shift={(2,0)}]
  \filldraw[box, fill=blue!20] (1,4.5) rectangle (1.25,4.75);
  \node[anchor=west] at (1.2,4.65) {Ação};
\end{scope}
\begin{scope}[shift={(3.3,0)}]
  \filldraw[box, fill=yellow!20] (1,4.5) rectangle (1.25,4.75);
  \node[anchor=west] at (1.2,4.65) {Replicação};
\end{scope}
\end{scope}

\end{scope}
\end{tikzpicture}

\end{frame}

% ==============================================================
\begin{frame}{Execução de Ação em Treplica}

\tikzset{
  every picture/.append style={
    transform shape, scale=0.67
  }
}

\begin{sequencediagram}[font=\fontsize{0.28cm}{0.31cm}\selectfont\ttfamily,scale=0.5]
	\draw (5.6,-1)  node[right] {\small Réplica A};
	\draw (8.3,-1)  node[right] {\small Réplica B};
	\draw (8,0) -- (8,-4.8);
    \draw (8,-5.8) -- (8,-10.3);
	%\draw (8,-5.8) -- (8,-8.8);
	%\draw (8,-9.5) -- (8,-10.3);
	\newthread{InstAMain}{Program}
	\newinst{InstAInfo}{Info}
	\newinst{InstAUpdate}{UpdateAction}
	\newinst{InstAState}{Treplica}
	
	\newinst[0.6]{InstBState}{Treplica}
	\newinst{InstBUpdate}{UpdateAction}
	\newinst{InstBInfo}{Info}
	\newthread{InstBMain}{Program}
	
	\begin{call}{InstAMain}{Info new}{InstAInfo}{info} \end{call}
    \begin{call}{InstAMain}{Treplica new}{InstAState}{treplica} \end{call}
	\begin{call}{InstAMain}{runMachine:}{InstAState}{} \end{call}
	
	\prelevel \prelevel \prelevel \prelevel \prelevel \prelevel
	
	\begin{call}{InstBMain}{Info new}{InstBInfo}{info} \end{call}
	\begin{call}{InstBMain}{Treplica new}{InstBState}{treplica} \end{call}
	\begin{call}{InstBMain}{runMachine:}{InstBState}{} \end{call}	
        
	\begin{call}{InstAMain}{execute:}{InstAState}{}
		\postlevel
		\begin{call}{InstAState}{executeOn:}{InstAUpdate}{} 
			\begin{call}{InstAUpdate}{setText:}{InstAInfo}{} \end{call}
		\end{call}
	
		\prelevel \prelevel \prelevel 
		\prelevel \prelevel  
	
		\begin{messcall}{InstAState}{replicação}{InstBState}{}
			\postlevel
			\begin{call}{InstBState}{executeOn:}{InstBUpdate}{} 
				\begin{call}{InstBUpdate}{setText:}{InstBInfo}{} \end{call}
			\end{call}
		\end{messcall}
	\end{call}
\end{sequencediagram}

\end{frame}


% ==============================================================
\begin{frame}{Metaprogramação em Cyan}

\justifying
A metaprogramação é a criação de um código-fonte (programa) que reflete seu significado sobre ele próprio. Em Cyan metaprogramação é realizada em tempo de compilação usando metaobjetos.

\vspace{15px}

\begin{tikzpicture}[yscale=-1]

\node[draw=black!60, fill=red!20, thick, anchor=north west, align=center] at (0,1) {Código-Fonte Cyan\\com @anotações};
\node[draw=black!60, fill=green!20, thick, anchor=north west, align=center] at (4.25,1) {Metaobjetos das\\@anotações};
\node[draw=black!60, fill=gray!20, thick, anchor=north west, align=center] at (0,3) {Compilação\\Etapa A};
\node[draw=black!60, fill=red!20, thick, anchor=north west, align=center] at (4,3) {Nova versão do\\Código-Fonte Cyan};
\node[draw=black!60, fill=gray!20, thick, anchor=north west, align=center] at (8.78,3) {Compilação\\Etapa B};
\node[draw=black!60, fill=red!20, thick, anchor=north west, align=center] at (8.5,5) {Código-Fonte\\Java};

\draw[->, thick, draw=black!70] (0.95,2.05) to (0.95, 2.98);
\draw[->, thick, draw=black!70] (5.5,2.05) .. controls (5.5,3) and (1.05,2) .. (1.1, 2.98);
\draw[->, thick, draw=black!70] (1.95,3.5) to (3.98, 3.5);
\draw[->, thick, draw=black!70] (7.08,3.5) to (8.77, 3.5);
\draw[->, thick, draw=black!70] (9.7,4.05) to (9.7, 4.99);

\begin{scope}[shift={(-0.9,1.15)}]
\begin{scope}[shift={(0,0)}]
  \filldraw[box, fill=red!20] (1,4.5) rectangle (1.25,4.75);
  \node[anchor=west] at (1.2,4.65) {Código-Fonte};
\end{scope}
\begin{scope}[shift={(2.95,0)}]
  \filldraw[box, fill=green!20] (1,4.5) rectangle (1.25,4.75);
  \node[anchor=west] at (1.2,4.65) {Metaobjeto};
\end{scope}
\begin{scope}[shift={(5.7,0)}]
  \filldraw[box, fill=gray!20] (1,4.5) rectangle (1.25,4.75);
  \node[anchor=west] at (1.2,4.65) {Compilador};
\end{scope}
\end{scope}

\end{tikzpicture}

\end{frame}

% ==============================================================
\begin{frame}{Treplica usando Metaobjetos}

\begin{tikzpicture}[yscale=-1]
\begin{scope}[shift={(0,1)}]

\begin{scope}[shift={(4.385,0.020)}]
  \filldraw[draw=black!60, thick, fill=red!20] (0,0) -- (0.8,0) -- (0.8,0.8) -- (0,0) -- cycle;
\end{scope}

\begin{scope}
\node[draw=black!60, thick, anchor=north west, align=left, font=\footnotesize, minimum width=5.174cm] at (0,0) {
package main \\
import treplica \\
\\
object Program \\
~~func run: Array<String> args \{ \\
~~~~var local = "/tmp" ++ args[1]; \\
~~~~{\color{britishracinggreen!67}@treplicaInit}( 3, 200 , local ) \\
~~~~var info = Info new; \\
~~~~if args[1] == "ReplicaA" \{ \\
~~~~~~info setText: "text"; \\
~~~~\} \\
~~\} \\
end
};
\end{scope}

\begin{scope}[shift={(10.01,0.020)}]
  \filldraw[draw=black!60, thick, fill=red!20] (0,0) -- (0.8,0) -- (0.8,0.8) -- (0,0) -- cycle;
\end{scope}

\begin{scope}[shift={(6.3,0)}]
\node[draw=black!60, thick, anchor=north west, align=left, font=\footnotesize, minimum size=4.5cm] at (0,0) {
package main \\
import treplica \\
\\
object Info extends Context \\
~~var String text \\
~~func init \{ self.text = "~";\} \\
\\
~~{\color{britishracinggreen!67}@treplicaAction} \\
~~func setText: String text \{ \\
~~~~self.text = text; \\
~~\} \\
\\
~~func getText -> String \{ \\
~~~~return self.text; \\
~~\} \\
end
};
\end{scope}

\begin{scope}[scale=0.8, shift={(1.8,5.25)}]
  \filldraw[box, fill=red!20] (1,1.5) rectangle (2,2.5);
  \filldraw[box, fill=blue!20] (1,2) rectangle (1.5,2.5);
  \filldraw[box, fill=yellow!20] (1.5, 2) rectangle (2,2.5);
\end{scope}

\filldraw[fill=black!70, draw=black!70] (2.45,6) circle (1.5pt);
\filldraw[fill=black!70, draw=black!70] (2.85,6) circle (1.5pt);

\draw[very thick, draw=orange!47] (0.75,2.05) to (0.75,3.1);
\draw[->, thick, draw=black!70] (2.85,6) to[bend right=68] (0.7,3.1);

\draw[very thick, draw=orange!47] (8.1,1.55) to (10.25,1.55);
\draw[->, thick, draw=black!70] (2.85,6) to[bend left=65] (9.8,1.6);

\draw[very thick, draw=blue!45] (0.9,3.9) to (3.7,3.9);
\draw[->, thick, draw=black!70] (2.45,6.05) to (2.45,4);

\draw[very thick, draw=blue!45] (6.85,2.79) to (6.85,4.3);
\draw[->, thick, draw=black!70] (2.45,6.05) to[bend right] (6.75, 3.15);

\begin{scope}[shift={(-0.9,2.1)}]
\begin{scope}[shift={(0,0)}]
  \filldraw[box, fill=red!20] (1,4.5) rectangle (1.25,4.75);
  \node[anchor=west] at (1.2,4.65) {Programa};
\end{scope}
\begin{scope}[shift={(2,0)}]
  \filldraw[box, fill=blue!20] (1,4.5) rectangle (1.25,4.75);
  \node[anchor=west] at (1.2,4.65) {Ação};
\end{scope}
\begin{scope}[shift={(3.3,0)}]
  \filldraw[box, fill=yellow!20] (1,4.5) rectangle (1.25,4.75);
  \node[anchor=west] at (1.2,4.65) {Replicação};
\end{scope}
\begin{scope}[shift={(5.45,0)}]
  \filldraw[box, fill=green!20] (1,4.5) rectangle (1.25,4.75);
  \node[anchor=west] at (1.2,4.65) {Anotação};
\end{scope}
\end{scope}

\end{scope}
\end{tikzpicture}

\end{frame}

% ==============================================================
% \colorbox{blue!30}{blue}
% \textcolor{blue!30}{blue}
\begin{frame}{@treplicaInit}

\setlength\fboxsep{1pt}
\begin{tikzpicture}[yscale=-1]

\begin{scope}
\node[draw=black!60, thick, anchor=north west, align=left, font=\scriptsize, minimum width=4.5cm] at (0,0) {
package main \\
import treplica \\
\\
object Program \\
~~func run: Array<String> args \{ \\
~~~~var local = "/tmp" ++ args[1]; \\
\\
~~~~{\color{britishracinggreen!67}@treplicaInit}( \colorbox{red!20}{3}, \colorbox{red!20}{200} , \colorbox{red!20}{local} ) \\
~~~~var \colorbox{blue!20}{info} = Info new; \\
\\
~~~~if args[1] == "ReplicaA" \{ \\
~~~~~~info setText: "text"; \\
~~~~\} \\
~~\} \\
end
};
\end{scope}

\node[draw=black!60, thick, anchor=north west, align=center, font=\scriptsize, minimum width=4.5cm] at (0,4.995) {Código-Fonte Original};
\node[draw=black!60, thick, anchor=north west, align=center, font=\scriptsize, minimum width=4.547cm] at (6.3,6.29) {Código-Fonte Gerado};

\begin{scope}[shift={(6.3,0)}]
\node[draw=black!60, thick, anchor=north west, align=left, font=\scriptsize] at (0,0) {
package main \\
import treplica \\
\\
object Program \\
~~func run: Array<String> args \{ \\
~~~~var local = "/tmp" ++ args[1]; \\
\\
~~~~var info = Info new; \\
~~~~var treplica\colorbox{blue!20}{info} = Treplica new; \\
~~~~treplica\colorbox{blue!20}{info} runMachine: \\
~~~~~~\colorbox{blue!20}{info} numberProcess: \colorbox{red!20}{3} rtt: \colorbox{red!20}{200} \\
~~~~~~path: \colorbox{red!20}{local}; \\
~~~~\colorbox{blue!20}{info} setTreplica: treplica\colorbox{blue!20}{info}; \\
\\
~~~~if args[1] == "ReplicaA" \{ \\
~~~~~~info setText: "text"; \\
~~~~\} \\
~~\} \\
end
};
\end{scope}

\draw[very thick, draw=orange!47] (0.6,2.28) to (4.2,2.28);
\draw[very thick, draw=orange!47] (0.6,3.02) to (4.2,3.02);

\draw[very thick, draw=orange!47, dashed] (4.2,2.28) to (6.75,2.28);
\draw[very thick, draw=orange!47, dashed] (4.2,3.02) to (6.75,4.3);

\draw[very thick, draw=orange!47] (6.75,2.28) to (10.5,2.28);
\draw[very thick, draw=orange!47] (6.75,4.3) to (10.5,4.3);

\node[draw=black!60, fill=gray!20, thick, anchor=north west, align=center] at (1.25,5.8) {Compilação\\Etapa A};
\draw[->, thick, draw=black!70] (3.2,6.4) to[bend left=35] (5.5,3.1);

\end{tikzpicture}

\end{frame}

% ==============================================================
\begin{frame}{@treplicaAction}

\setlength\fboxsep{1pt}
\begin{tikzpicture}[yscale=-1]

\begin{scope}[shift={(0,0)}]
\node[draw=black!60, thick, anchor=north west, align=left, font=\scriptsize] at (0,0) {
... \\
object \colorbox{red!20}{Info} extends Context \\
... \\
~~{\color{britishracinggreen!67}@treplicaAction} \\
~~func \colorbox{blue!20}{setText}: \colorbox{violet!20}{String} \colorbox{black!20}{text} \{ \\
~~~~self.text = text; \\
~~\} \\
... \\
end
};
\end{scope}

\begin{scope}[shift={(0,3.5)}]
\node[draw=black!60, thick, anchor=north west, align=left, font=\scriptsize] at (0,0) {
... \\
object Info extends Context \\
... \\
~~func \colorbox{blue!20}{setText}: \colorbox{violet!20}{String} \colorbox{black!20}{text} \{\\
~~~~var action = \colorbox{red!20}{Info}\colorbox{blue!20}{setText} new: \colorbox{black!20}{text}; \\
~~~~self getTreplica execute: action; \\
~~\} \\
\\
~~func \colorbox{blue!20}{setText}TreplicaAction: \colorbox{violet!20}{String} \colorbox{black!20}{text} \{ \\
~~~~self.text = text; \\
~~\} \\
... \\
end
};
\end{scope}

\begin{scope}[shift={(5.8,0)}]
\node[draw=black!60, thick, anchor=north west, align=left, font=\scriptsize] at (0,0) {
package main \\
import treplica \\
\\
object \colorbox{red!20}{Info}\colorbox{blue!20}{setText} extends Action \\
~~var \colorbox{violet!20}{String} \colorbox{black!20}{text}Var \\
\\
~~func init: \colorbox{violet!20}{String} \colorbox{black!20}{text} \{ \\
~~~~self.\colorbox{black!20}{text}Var = \colorbox{black!20}{text}; \\
~~\} \\
\\
~~func executeOn: Context context \{ \\
~~~~var obj = \colorbox{red!20}{Info} cast: context; \\
~~~~obj \colorbox{blue!20}{setText}TreplicaAction: self.\colorbox{black!20}{text}Var; \\
~~\} \\
end
};
\end{scope}

\node[draw=black!60, thick, anchor=north west, align=center, font=\scriptsize, minimum width=3.68cm] at (0,2.9) {Código-Fonte Original};
\node[draw=black!60, thick, anchor=north west, align=center, font=\scriptsize, minimum width=5.313cm] at (0,7.707) {Código-Fonte Gerado};
\node[draw=black!60, thick, anchor=north west, align=center, font=\scriptsize, minimum width=5.2cm] at (5.8,5) {Código-Fonte Gerado};

\node[draw=black!60, fill=gray!20, thick, anchor=north west, align=center] at (7.5,6.5) {Compilação\\Etapa A};

\draw[->, very thick, draw=orange!47] (3.75,1) to (5.75,1);

\draw[very thick, draw=orange!47] (0.2,6.32) to (5,6.32);
\draw[->, very thick, draw=orange!47] (3.75,1.75) to[bend right=45] (4.7,6);

\draw[very thick, draw=orange!47] (0.2,4.4) rectangle (4.7,5.7);
\draw[->, very thick, draw=orange!47] (3.75,2.5) to[bend right=45] (4.3,4.3);

\draw[very thick, draw=orange!47, dashed] (3.75,0.5) to (5.75,0.5);
\draw[very thick, draw=orange!47, dashed] (3.75,3) to (5.75,3);

\draw[->, thick, draw=black!70] (7.5,7) to[bend right=28] (5.4,2.5);

\end{tikzpicture}

\end{frame}

% ==============================================================
\begin{frame}{Restrições e Verificações}

\setlength\fboxsep{1pt}
\begin{tikzpicture}[yscale=-1]

\begin{scope}[shift={(0,0)}]
\node[draw=black!60, thick, anchor=north west, align=left, font=\footnotesize, minimum size=4.5cm] at (0,0) {
package \colorbox{black!20}{main} \\
... \\
object \colorbox{red!20}{Info} extends Context \\
~~... \\
~~{\color{britishracinggreen!67}@treplicaAction} \\
~~func setText: String text \{ \\
~~~~var num = \colorbox{blue!20}{getRandom}; \\
~~~~self.text = text ++ num; \\
~~\} \\
\\
~~func \colorbox{blue!20}{getRandom} -> String \{ \\
~~~~return Math random; \\
~~\} \\
\\
~~func \colorbox{violet!20}{getFix} -> String \{ \\
~~~~return "23"; \\
~~\} \\
~~... \\
end
};
\end{scope}

\begin{scope}[shift={(5.4,1.7)}]
\node[draw=black!60, thick, anchor=north west, align=left, font=\footnotesize, minimum width=5.5cm] at (0,0) {
... \\
func setText: String text \{ \\
~~var num = \colorbox{black!20}{main}.\colorbox{red!20}{Info} \colorbox{violet!20}{getFix}; \\
~~self.text = text ++ num; \\
\} \\
...
};
\end{scope}

\begin{scope}[shift={(5.4,0)}]
\node[draw=black!60, thick, anchor=north west, align=left, font=\footnotesize] at (0,0) {
\colorbox{black!20}{main},\colorbox{red!20}{Info},\colorbox{blue!20}{getRandom}-\colorbox{black!20}{main},\colorbox{red!20}{Info},\colorbox{violet!20}{getFix}
};
\end{scope}

\node[draw=black!60, fill=gray!20, thick, anchor=north west, align=center] at (7.4,5.5) {Compilação\\Etapa A};
\draw[->, thick, draw=black!70] (7.4,6.1) to[bend right=35] (5,2.8);

\draw[very thick, draw=orange!47] (0.4,2.35) to (4.5,2.35);
\draw[very thick, draw=orange!47, dashed] (4.5,2.35) to (5.4,2.35);
\draw[very thick, draw=orange!47] (5.4,2.35) to (10.2,2.35);

\draw[very thick, draw=orange!47] (0.4,3.15) to (4.5,3.15);
\draw[very thick, draw=orange!47, dashed] (4.5,3.15) to (5.4,3.15);
\draw[very thick, draw=orange!47] (5.4,3.15) to (10.2,3.15);

\end{tikzpicture}

\end{frame}

% ==============================================================

{\setbeamercolor{palette primary}{fg=black, bg=yellow}
\begin{frame}[standout]
  Conclusão\\e\\Questões?
\end{frame}
}

\end{document}
