%================================= Resumo e Abstract ========================================
\setlength{\absparsep}{18pt} % ajusta o espaçamento dos paragrafos do resumo 
                     
% --- 
% resumo em português
% --- 
\begin{resumo}
\setlength\parfillskip{0pt plus .75\textwidth}
\setlength\emergencystretch{1pt}
O objetivo dessa pesquisa é mostrar como aplicações distribuídas, coesas e pouco acopladas podem ser desenvolvidas. A proposta é usar técnicas de metaprogramação em compilação para automatizar parte do desenvolvimento do código-fonte das aplicações e inspecionar esse código-fonte à procura de inconsistências. Para isso foi usado Treplica e Cyan. Treplica fornece uma estrutura para o desenvolvimento de aplicações distribuídas, enquanto a linguagem de programação Cyan provê suporte à metaprogramação. Esses recursos permitiram a criação de componentes que podem ser usados para desenvolver aplicações replicadas. Nessa pesquisa foi possível aplicar metaprogramação para automatizar etapas do desenvolvimento dessas aplicações, tornando o código-fonte da replicação melhor isolado do restante da aplicação. A verificação desse código-fonte na busca de inconsistências também pode ser demonstrada. Essa pesquisa não tem a pretensão de tratar todas as possibilidades de automatização do desenvolvimento do código-fonte replicado, e também não pretende tratar todas as verificações necessárias à replicação. O objetivo é mostrar que, usando metaprogramação em compilação, é possível automatizar o desenvolvimento e a inspeção de código-fonte das aplicações de modo geral.

\textbf{Palavras-chaves}: Replicação. Metaprogramação. Linguagens de Programação.

\end{resumo}
% --- 

\begin{resumo}[Abstract]
 \begin{otherlanguage*}{english}

The objective of this research is to show how distributed, cohesive and coupled applications can be developed. The proposal is to use metaprogramming technics in compilation to automate part of the applications source code development, and to inspect the source code to find inconsistencies. For this purpose Treplica and Cyan were used. Treplica provides a structure to the distributed applications development, while Cyan provides the support to metaprogramming. These resources allowed the components creation, which can be used to develop replicated applications. In this research it was possible to apply metaprogramming to automate developmento steps of these applications, making the source code of replication better isolated from the rest of the application. The verification of this source code to find inconsistencies can also be made. This research does not intend to solve all automation possibilities of the replicated source code development, and it does not intend to solve all necessary verifications to the replication either. The objective of this research is to show that, using metaprogramming, it is possible to automate the development and the inspection of the applications source code in general.   

\textbf{Keywords}: Replication. Metaprogramming. Programming Languages.

 \end{otherlanguage*}
\end{resumo}
% --- 

%===========================================================================================
